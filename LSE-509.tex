\documentclass[SE,toc]{lsstdoc}
% lsstdoc documentation: https://lsst-texmf.lsst.io/lsstdoc.html

% Generated by Makefile
\input{meta}

% Package imports go here.

% Local commands go here.

% If you want glossaries, uncomment:
% \input{aglossary.tex}
% \makeglossaries

\title{SIT-Com Management Plan}
% \setDocSubtitle{Optional subtitle}

\author{%
Patrick Ingraham
}

\setDocRef{LSE-509}
\setDocUpstreamLocation{\url{https://github.com/lsst-sitcom/LSE-509}}
\date{\vcsDate}
% \setDocCurator{The Curator of this Document - I AM THE CURATOR!}

\setDocAbstract{%
Add abstract text.
}

% Revision history.
% Order: oldest first.
% Fields: VERSION, DATE, DESCRIPTION, OWNER NAME.
% See LPM-51 for version number policy.
\setDocChangeRecord{%
  \addtohist{1}{2021-09-16}{Unreleased.}{Patrick Ingraham}
}

\begin{document}

\maketitle

% ADD CONTENT HERE
% You can also use the \input command to include several content files.

\appendix

% Include all the relevant bib files.
% https://lsst-texmf.lsst.io/lsstdoc.html#bibliographies
\section{References} \label{sec:bib}
\renewcommand{\refname}{} % Suppress default Bibliography section
\bibliography{local,lsst,lsst-dm,refs_ads,refs,books}

% Make sure lsst-texmf/bin/generateAcronyms.py is in your path
\section{Acronyms} \label{sec:acronyms}
\addtocounter{table}{-1}
\begin{longtable}{p{0.145\textwidth}p{0.8\textwidth}}\hline
\textbf{Acronym} & \textbf{Description}  \\\hline

AIV & Assembly Integration and Verification \\\hline
ComCam & The commissioning camera is a single-raft, 9-CCD camera that will be installed in LSST during commissioning, before the final camera is ready. \\\hline
DM & Data Management \\\hline
DOE & Department of Energy \\\hline
EFD & Engineering and Facility Database \\\hline
EVMS & Earned Value Management System \\\hline
FRACAS & Failure Reporting Analysis and Corrective Action System \\\hline
HSC & Hyper Suprime-Cam \\\hline
IN2P3 & Institut National de Physique Nucléaire et de Physique des Particules \\\hline
IT & Information Technology \\\hline
LCA & Document handle LSST camera subsystem controlled documents \\\hline
LCR & LSST Change Request \\\hline
LPM & LSST Project Management (Document Handle) \\\hline
LSE & LSST Systems Engineering (Document Handle) \\\hline
LSR & LSST System Requirements; LSE-29 \\\hline
LSST & Legacy Survey of Space and Time (formerly Large Synoptic Survey Telescope) \\\hline
LVV & LSST Verification and Validation \\\hline
M1M3 & Primary Mirror Tertiary Mirror \\\hline
MIE & Major Item of Equipment \\\hline
MREFC & Major Research Equipment and Facility Construction \\\hline
NSF & National Science Foundation \\\hline
OCS & Observatory Control System \\\hline
OSS & Observatory System Specifications; LSE-30 \\\hline
PEP & Project Execution Plan \\\hline
PO & Program Operations \\\hline
PSE & Project Systems Engineering \\\hline
QA & Quality Assurance \\\hline
SE & System Engineering \\\hline
SIT & System Integration, Test \\\hline
SIT-COM & System Integration, Test and Commissioning \\\hline
SV & Science Validation \\\hline
T\&S & Telescope and Site \\\hline
TBD & To Be Defined (Determined) \\\hline
TBR & To Be Resolved \\\hline
TCS & Telescope Control System \\\hline
TMA & Telescope Mount Assembly \\\hline
US & United States \\\hline
WBS & Work Breakdown Structure \\\hline
\end{longtable}

% If you want glossary uncomment below and comment out the two lines above.
% \printglossaries

\end{document}
